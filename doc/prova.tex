%%%%%%%%%%%%%%%%%%%%%%%%%%%%%%%%%%%%%%%%%%%%%%%%%%%%%%%%%%%%%%%%%%%%%%
% LaTeX Example: Project Report
%
% Source: http://www.howtotex.com
%
% Feel free to distribute this example, but please keep the referral
% to howtotex.com
% Date: March 2011
%
%%%%%%%%%%%%%%%%%%%%%%%%%%%%%%%%%%%%%%%%%%%%%%%%%%%%%%%%%%%%%%%%%%%%%%
% How to use writeLaTeX:
%
% You edit the source code here on the left, and the preview on the
% right shows you the result within a few seconds.
%
% Bookmark this page and share the URL with your co-authors. They can
% edit at the same time!
%
% You can upload figures, bibliographies, custom classes and
% styles using the files menu.
%
% If you're new to LaTeX, the wikibook is a great place to start:
% http://en.wikibooks.org/wiki/LaTeX
%
%%%%%%%%%%%%%%%%%%%%%%%%%%%%%%%%%%%%%%%%%%%%%%%%%%%%%%%%%%%%%%%%%%%%%%
% Edit the title below to update the display in My Documents
%\title{Project Report}
%
%%% Preamble
\documentclass[paper=a4, fontsize=11pt]{scrartcl}
\usepackage[T1]{fontenc}
\usepackage{fourier}

\usepackage[english]{babel}															% English language/hyphenation
\usepackage[protrusion=true,expansion=true]{microtype}
\usepackage{amsmath,amsfonts,amsthm} % Math packages
\usepackage[pdftex]{graphicx}
\usepackage{url}


%%% Custom sectioning
\usepackage{sectsty}
\allsectionsfont{\centering \normalfont\scshape}


%%% Custom headers/footers (fancyhdr package)
\usepackage{fancyhdr}
\usepackage{tikz}
\usetikzlibrary{automata, positioning, arrows}
\pagestyle{fancyplain}
\fancyhead{}											% No page header
\fancyfoot[L]{}											% Empty
\fancyfoot[C]{}											% Empty
\fancyfoot[R]{\thepage}									% Pagenumbering
\renewcommand{\headrulewidth}{0pt}			% Remove header underlines
\renewcommand{\footrulewidth}{0pt}				% Remove footer underlines
\setlength{\headheight}{13.6pt}


%%% Equation and float numbering
\numberwithin{equation}{section}		% Equationnumbering: section.eq#
\numberwithin{figure}{section}			% Figurenumbering: section.fig#
\numberwithin{table}{section}				% Tablenumbering: section.tab#


%%% Maketitle metadata
\newcommand{\horrule}[1]{\rule{\linewidth}{#1}} 	% Horizontal rule

\title{
%\vspace{-1in}
    \usefont{OT1}{bch}{b}{n}
    \includegraphics[scale=0.7]{Images/logo_polimi_scritta2.eps}
    \horrule{0.5pt} \\[0.4cm]
    \huge Prova finale di Reti Logiche \\
    \horrule{2pt} \\
    [0.1cm]
}
\author{
    \normalfont
    Simone Corbo,
    \normalsize{Matricola 955854}, \normalsize{Codice Persona 10727140}\\
}
\date{}

\tikzset{%-&gt;,  % makes the edges directed
    &gt;=stealth', % makes the arrow heads bold stealth
    node distance=3cm, % specifies the minimum distance between two nodes. Change if necessary.
    every state/.style={thick, fill=bln\_blue!10}, % sets the properties for each ’state’ node
    initial text=$\ \ $, % sets the text that appears on the start arrow
}


%%% Begin document
\begin{document}
    \maketitle
    \section{Introduzione}
    \label{section:intro}
    L'obiettivo non è la "copia" della specifica ma una elaborazione, con un esempio e, se è possibile, un disegno e/o una immagine, che spieghi cosa succede;
\begin{figure}[ht]
    \label{FSM}
    \begin{tikzpicture}
        \node[state, initial] (q1) {$q_1$};
        \node[state, right of=q1] (q2) {$q_2$};
        \node[state, right of=q2] (q3) {$q_3$};
        \draw (q1) edge[loop above] node{0} (q1)
        (q1) edge[above] node{1} (q2)
        (q2) edge[loop above] node{1} (q2)
        (q2) edge[bend left, above] node{0} (q3)
        (q3) edge[bend left, below] node{0, 1} (q2);
    \end{tikzpicture}


    \begin{tikzpicture}
        \node[state, initial, initial where=left] (1) {$ $};
        \node[state, right of=1] (2) {$ $};
        \node[state, right of=2] (3) {$ $};
        \node[state, right of=3] (4) {$ $};
        \node[state, below right of=3] (5) {$ $};
        \node[state, below left of=3] (6) {$ $};
        \node[state, right of=5] (7) {$ $};
        \draw
        % State names
        (1) node[above= 0.5,text width=1cm] {$Initial$}
        (2) node[above= 0.5,text width=1cm] {$Pending$}
        (3) node[above= 0.5,text width=1cm] {$Closed$}
        (4) node[above= 0.5,text width=1cm] {$Suceess$}
        (5) node[above= 0.5,text width=1cm] {$Open$}
        (6) node[above= 0.5,text width=1cm ] {$Half\-Open$}
        (7) node[above= 0.5,text width=1cm] {$Failure$}
        % State relations
        (1) edge[below] node[scale=0.8,text width=1.2cm]{create request} (2)
        (2) edge[below] node[scale=0.8,text width=1.2cm]{execute request} (3)
        (3) edge[above] node[scale=0.7]{request succeeded} (4)
        (3) edge[above right = 0.1] node[scale=0.7,text width=1cm]{request failed} (5)
        (5) edge[above, bend right] node[scale=0.8]{timeout reached} (6)
        (6) edge[below, bend right] node[scale=0.8]{canary failed} (5)
        (5) edge[below] node[scale=0.8]{timeout not reached} (7)
        (6) edge[above left= 1] node[scale=0.7,text width=1.7cm]{canary succeeded} (3)
        ;
    \end{tikzpicture}
\end{figure}



    \section{Architettura}
    \label{sec:arch}
        L’obiettivo è quello di riportare un schema funzionale (lo schema in moduli... un bel disegno chiaro... i segnali i bus, il/i clock, reset... );
    \subsection{Modulo 1}
    \label{subsec:mod1}
    (la descrizione - sottoparagrafo - di ogni modulo e la scelta
    implementativa - per esempio, il modulo ... è una collezione di process che
    implementano la macchina a stati e la parte di registri, .... La macchina a stati,
    il cui schema in termini di diagramma degli stati, ha 8 stati. Il primo
    rappresenta .... e svolge le operazioni di ... il secondo... etc etc)
    \subsection{Modulo ...}
    \label{subsec:mod}

    \section{Risultati sperimentali}
    \label{sec:result}
    \subsection{Sintesi (Report di sintesi)}
    \label{subsec:report}
    \subsection{Simulazioni}
    \label{subsec:sim}
    L'obiettivo non è solo riportare i risultati ottenuti attraverso la
    simulazione test bench forniti dai docenti, ma anche una analisi personale e
    una identificazione dei casi particolari; il fine è mostrare in modo convincente
    e più completo possibile, che il problema è stato esamintato a fondo e che,
    quanto sviluppato, soddisfa completamente i requisiti.
    i. test bench 1 (cosa fa e perchè lo fa e cosa verifica; per esempio,
    controlla una condizione limite)
    ii. test bench 2 (....)
    4. conclusioni (mezza pagina max
%
%    \section{Heading on level 1 (section)}
%    Lorem ipsum dolor sit amet, consectetuer adipiscing elit. Aenean commodo ligula eget dolor. Aenean massa. Cum sociis natoque penatibus et magnis dis parturient montes, nascetur ridiculus mus. Donec quam felis, ultricies nec, pellentesque eu, pretium quis, sem. In enim justo, rhoncus ut, imperdiet a, venenatis vitae, justo. Nullam dictum felis eu pede mollis pretium. Integer tincidunt. Cras dapibus. Vivamus elementum semper nisi. Aliquam lorem ante, dapibus in, viverra quis, feugiat a, tellus:
%    \begin{align}
%        \begin{split}
%        (x+y)^3 	&= (x+y)^2(x+y)\\
%        &=(x^2+2xy+y^2)(x+y)\\
%        &=(x^3+2x^2y+xy^2) + (x^2y+2xy^2+y^3)\\
%        &=x^3+3x^2y+3xy^2+y^3
%        \end{split}
%    \end{align}
%    Phasellus viverra nulla ut metus varius laoreet. Quisque rutrum. Aenean imperdiet. Etiam ultricies nisi vel augue. Curabitur ullamcorper ultricies
%
%    \subsection{Heading on level 2 (subsection)}
%    Lorem ipsum dolor sit amet, consectetuer adipiscing elit.
%    \begin{align}
%        A =
%        \begin{bmatrix}
%            A_{11} & A_{21} \\
%            A_{21} & A_{22}
%        \end{bmatrix}
%    \end{align}
%    Aenean commodo ligula eget dolor. Aenean massa. Cum sociis natoque penatibus et magnis dis parturient montes, nascetur ridiculus mus. Donec quam felis, ultricies nec, pellentesque eu, pretium quis, sem.
%
%    \subsubsection{Heading on level 3 (subsubsection)}
%    Nulla consequat massa quis enim. Donec pede justo, fringilla vel, aliquet nec, vulputate eget, arcu. In enim justo, rhoncus ut, imperdiet a, venenatis vitae, justo. Nullam dictum felis eu pede mollis pretium. Integer tincidunt. Cras dapibus. Vivamus elementum semper nisi. Aenean vulputate eleifend tellus. Aenean leo ligula, porttitor eu, consequat vitae, eleifend ac, enim.
%
%    \paragraph{Heading on level 4 (paragraph)}
%    Lorem ipsum dolor sit amet, consectetuer adipiscing elit. Aenean commodo ligula eget dolor. Aenean massa. Cum sociis natoque penatibus et magnis dis parturient montes, nascetur ridiculus mus. Donec quam felis, ultricies nec, pellentesque eu, pretium quis, sem. Nulla consequat massa quis enim.
%
%
%    \section{Lists}
%
%    \subsection{Example for list (3*itemize)}
%    \begin{itemize}
%        \item First item in a list
%        \begin{itemize}
%            \item First item in a list
%            \begin{itemize}
%                \item First item in a list
%                \item Second item in a list
%            \end{itemize}
%            \item Second item in a list
%        \end{itemize}
%        \item Second item in a list
%    \end{itemize}
%
%    \subsection{Example for list (enumerate)}
%    \begin{enumerate}
%        \item First item in a list
%        \item Second item in a list
%        \item Third item in a list
%    \end{enumerate}
%%% End document
\end{document}